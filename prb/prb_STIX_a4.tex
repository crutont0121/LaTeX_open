%% prb :: version 2
%% japanese font :: Toppan Bunkyu Min
%% english font :: STIX
%% type :: LuaLaTeX

\documentclass[a4paper]{ltjsarticle}

\usepackage{geometry}
  \geometry{margin=3cm, top=2.5cm}
\usepackage{amsmath, amssymb, amsthm}
\usepackage{tikz}

\usepackage{ascmac} % boxnote, itembox, mask, screen, shadebox
\usepackage{comment}
\usepackage{empheq}
\usepackage[shortlabels, inline]{enumitem}
\usepackage{fancybox} % doublebox, ovalbox, fbox, sbox
\usepackage{fancyhdr}

\usepackage{here}
\usepackage{lastpage}
\usepackage{listings, jvlisting}

\usepackage{stmaryrd} % special symbols
\usepackage[listings]{tcolorbox}
\usepackage{titlesec}

\usepackage{ulem} %% uline, uuline, uwave, sout, xout
\usepackage{url} %% url
\usepackage{verbatim} %%\verb|hoge| or verbatim env. option[*] : space visible
\usepackage{wrapfig}
\usepackage[svgnames]{xcolor}
\usepackage{varwidth}
\usepackage{wrapstuff}

\usepackage{siunitx}
\usepackage{physics2}
\usephysicsmodule{ab}
\usephysicsmodule{braket}
\usephysicsmodule{diagmat}
\usephysicsmodule{doubleprod}
\usephysicsmodule{xmat}

\usepackage[ISO]{diffcoeff} %% detail setting in renew_newcmd_ver2.tex
\usepackage[math-style=TeX]{unicode-math}
\unimathsetup{math-style=TeX}
\setmathfont{latinmodern-math.otf}%default
\setmathfont{xcharter-math.otf}[range=\symbb]%blackboard bold
\usepackage{mathtools}
\usepackage[no-math,deluxe,expert]{luatexja-preset}

\usepackage[luatex,pdfencoding=auto]{hyperref}
\hypersetup{%
	colorlinks = true,%
	linkcolor = dark-mint,%
	%linkcolor = blue,%
	urlcolor = orange,%
	citecolor = Green,%
}


%% font settings

%\usepackage{CormorantGaramond}
%\setmathfont{Garamond-Math.otf}

%\setmainjfont{Harano Aji Mincho}[%
%	BoldFont = {Harano Aji Gothic Bold},
%	%BoldFont = {Harano Aji Mincho Bold},%
%	Scale = 0.87%
%]
\setmainjfont{Toppan Bunkyu Mincho Regular}[
	Scale=0.90
]

%\setmainfont[%
%ItalicFont=NewCM10-Italic.otf,%
%BoldFont=NewCM10-Bold.otf,%
%BoldItalicFont=NewCM10-BoldItalic.otf,%
%SmallCapsFeatures={Numbers=OldStyle}]{NewCM10-Regular.otf}
%\setmathfont{Theano Modern Regular}[range={"0030-"0039}]

%\setmainfont[%
%ItalicFont=NewCM10-Italic.otf,%
%BoldFont=NewCM10-Bold.otf,%
%BoldItalicFont=NewCM10-BoldItalic.otf,%
%SmallCapsFeatures={Numbers=OldStyle}]{NewCM10-Regular.otf}

%\setsansfont[%
%ItalicFont=NewCMSans10-Oblique.otf,%
%BoldFont=NewCMSans10-Bold.otf,%
%BoldItalicFont=NewCMSans10-BoldOblique.otf,%
%SmallCapsFeatures={Numbers=OldStyle}]{NewCMSans10-Regular.otf}

%\setmonofont[ItalicFont=NewCMMono10-Italic.otf,%
%BoldFont=NewCMMono10-Bold.otf,%
%BoldItalicFont=NewCMMono10-BoldOblique.otf,%
%SmallCapsFeatures={Numbers=OldStyle}]{NewCMMono10-Regular.otf}

%\setmathfont{NewCMMath-Regular.otf}
\setmathfont{STIXTwoMath.otf}
\setmainfont{STIXTwoText-Regular}[
    Extension=.otf,
    BoldFont=STIXTwoText-Bold,
    ItalicFont=STIXTwoText-Italic,
    BoldItalicFont=STIXTwoText-BoldItalic
]

\numberwithin{equation}{section}

%% TITLEFORMAT SETTING
%\titleformat{\section}[block]{\filleft\large}{\uline{\LARGE\thesection.\hspace{.5em}}}{0em}{\uline}
%\titleformat{\subsection}{\normalsize}{\uline{\large\bfseries\thesubsection.\hspace{.5em}}}{0em}{\uline}
%\titleformat{\subsubsection}{\normalsize}{\uline{\bfseries\thesubsubsection.}\hspace{.5em}}{0em}{}

\titleformat{\section}[block]{\filleft\large}{\uline{\LARGE\thesection.\hspace{.5em}}}{0em}{\uline}
\titleformat{\subsection}{\normalsize}{\uline{\Large\thesubsection.\hspace{.5em}}}{0em}{\uline}
\titleformat{\subsubsection}{\normalsize}{\uline{\large\thesubsubsection.}\hspace{.5em}}{0em}{}
\titleformat{\paragraph}{\normalfont\normalsize}{\theparagraph.}{1em}{}



\renewcommand{\thesection}{\arabic{section}}
\renewcommand{\thesubsection}{\arabic{section}.\arabic{subsection}}
\renewcommand{\thesubsubsection}{\arabic{section}.\arabic{subsection}.\arabic{subsubsection}}
%\renewcommand{\thesubsubsection}{\fontspec[Scale=0.8]{Futura Medium}\arabic{section}.\arabic{subsection}.\arabic{subsubsection}}
\titlespacing{\subsection}{0pt}{5pt}{5pt}
\titlespacing{\subsubsection}{0pt}{3pt}{5pt}
%\titlespacing*{\paragraph}{0pt}{3.25ex plus 1ex minus .2ex}{1.5ex plus .2ex}
\titlespacing*{\paragraph}{0pt}{3pt}{5pt}

%% THEOREMSTYLE
\theoremstyle{definition}
\newtheorem{dfn}{Definition}[section]
\newtheorem{prop}[dfn]{Proposition}
\newtheorem{lem}[dfn]{Lemma}
\newtheorem{thm}[dfn]{Theorem}
\newtheorem{cor}[dfn]{Corollary}
\newtheorem{rem}[dfn]{Remark}
\newtheorem{fact}[dfn]{Fact}
% TCBUSELIBRARY
\tcbuselibrary{most}
\tcbset{colback=white,colframe=blue!75!black}
\newtcbtheorem[number within = section]{example}{Example}{
    enhanced,
    %colback = white
    colframe = blue!50!white,
    %fonttitle=\bfseries,
    breakable = true
  }{thm}
  %enhanced, breakable=true, colframe=blue!75!black, colback=blue!5!white, title={Exemple}, arc = 0mm}
  %{colback=green!5,colframe=green!35black,fonttitle=\bfseries}{th}

\newtcbtheorem[number within = section]{exercise}{Exercise}{
  enhanced, 
  breakable=true,
  colframe=yellow!25!green
  %colback=white
  }{thm}

\newtcbtheorem[use counter from=example]{theorem}{Theorem}{
  enhanced, 
  breakable=true, 
  colframe=green!30!blue
  %colback=white
  }{thm}

\newtcbtheorem[use counter from=example]{proposition}{Proposition}{
  enhanced, 
  breakable=true, 
  colframe=orange!75!black,
  %colback=white
  }{thm}

\newtcbtheorem[use counter from=example]{definition}{Definition}{
  enhanced, 
  breakable=true, 
  colframe=magenta!30!green,
  %colback=white
}{thm}
\newtcbtheorem[use counter from=example]{remark}{Remark}{
  enhanced,
  breakable=true, 
  colframe=red!80!white, 
  %colback=white
}{thm}


%% TCBSET INPUT

%% ========== %%
%%  Tcolorbox %%
%% ========== %%
\tcbset{
  dashedbox/.style={
    breakable,
    enhanced,
    frame hidden,
    borderline={0.3pt}{0pt}{blue!50!white,dashed},
    %fontupper = \fontsize{9pt}{8pt}\selectfont,
    %before upper = {\everymath{\scriptstyle}},
    colback = white,
    arc = 0mm,
    outer arc = 0mm,
    boxsep = 0.5pt,
    leftrule=-6pt,
    rightrule=-6pt
  }
}
\tcbset{
  leftsolid/.style={
    breakable,
    enhanced,
    colback = white,
    frame hidden,
    %arc = 0mm,
    %outer arc = 0mm,
    boxsep = 0.5pt,
    leftrule=4pt,
    rightrule=-6pt,
    borderline north={0.3pt}{0pt}{orange!30!white},
    borderline east={0.3pt}{0pt}{orange!30!white},
    borderline south={0.3pt}{0pt}{orange!30!white},
    borderline west={10pt}{0pt}{orange!10!white}
  }
}
\tcbset{
  leftsolid2/.style={
    breakable,
    enhanced,
    colback = white,
    frame hidden,
    %arc = 0mm,
    %outer arc = 0mm,
    boxsep = 0.5pt,
    leftrule=4pt,
    rightrule=-6pt,
    borderline north={0.3pt}{0pt}{blue!30!white},
    borderline east={0.3pt}{0pt}{blue!30!white},
    borderline south={0.3pt}{0pt}{blue!30!white},
    borderline west={10pt}{0pt}{blue!10!white}
  }
}


%% ================ %%
%% renewcmd, newcmd %%
%% ================ %%


%\renewcommand{\baselinestretch}{0.85}
\renewcommand{\headrulewidth}{0.3pt} %% thickness of line (btwn header andy body).0pt = no line
\renewcommand{\figurename}{Fig.\,}
\renewcommand{\tablename}{Table\,}



\newcommand{\mctext}[1]{\mbox{\textcircled{\scriptsize{#1}}}}
\newcommand{\ctext}[1]{\textcircled{\scriptsize{#1}}}
\newcommand{\ds}{\displaystyle}
\newcommand{\comb}[2]{\left(\begin{matrix}#1\\#2\end{matrix}\right)}
\newcommand{\hs}{\hspace}
\newcommand{\qq}{\hspace{1em}}
\newcommand{\qqtext}[1]{\hspace{1em}\text{#1}\hspace{0.5em}}
\newcommand{\vs}{\vspace}
\newcommand{\emphvs}{\vspace{1em}\notag\\}
\newcommand{\ora}{\overrightarrow}
\newcommand{\oramr}[1]{\overrightarrow{\mathrm{#1}}}
\newcommand{\ol}{\overline}
\newcommand{\tri}{\triangle}
\newcommand{\mr}{\mathrm}
\newcommand{\mb}{\mathbb}
\newcommand{\mrvec}[1]{\overrightarrow{\mathrm{#1}}}
\newcommand{\itvec}{\overrightarrow}
%\newcommand{\bs}{\boldsymbol}
\newcommand{\bs}{\symbfit}%%symbfでdefine
\newcommand{\bsup}{\symbfup}

\newcommand{\ra}{\rightarrow}
\newcommand{\Ra}{\Rightarrow}
\newcommand{\lra}{\longrightarrow}
\newcommand{\Lra}{\Longrightarrow}
\newcommand{\la}{\leftarrow}
\newcommand{\La}{\Leftarrow}
\newcommand{\lla}{\longleftarrow}
\newcommand{\Lla}{\Longleftarrow}
\newcommand{\lr}{\leftrightarrow}
\newcommand{\llr}{\longleftrightarrow}
\newcommand{\Llr}{\Longleftrightarrow}
\newcommand{\mqty}[1]{\begin{matrix}#1\end{matrix}}
\newcommand{\avg}[1]{\left\langle{#1}\right\rangle}

\newcommand{\eval}[1]{\left.#1\right|}
\newcommand{\order}[1]{\mathcal{O}\ab(#1)}
\newcommand{\gr}{\nabla}
\newcommand{\di}{\nabla\cdot}
\newcommand{\ro}{\nabla\times}
\newcommand{\nb}{\nabla}



\renewcommand{\i}{\symup{i}}
\newcommand{\e}{\symup{e}}
\newcommand{\R}{\symbb{R}}
\newcommand{\N}{\symbb{N}}
\newcommand{\C}{\symbb{C}}
\newcommand{\al}{\alpha}
\newcommand{\be}{\beta}
\newcommand{\ga}{\gamma}
\newcommand{\de}{\delta}
\newcommand{\eps}{\varepsilon}
%\newcommand{\th}{\theta}


%% MATH OP.

\DeclareMathOperator{\GL}{GL}
\DeclareMathOperator{\SL}{SL}
\DeclareMathOperator{\diag}{diag}
\DeclareMathOperator{\tr}{tr}
\DeclareMathOperator{\sinc}{sinc}


%% COLOR DEFINITION

\definecolor{mycream}{HTML}{FEFAEA}
\definecolor{myvio}{HTML}{03155B}
\definecolor{blue-green}{rgb}{0.0, 0.87, 0.87}
\definecolor{bluegray}{rgb}{0.4, 0.6, 0.8}
\definecolor{mint}{rgb}{0.24, 0.71, 0.54}
\definecolor{dark-mint}{HTML}{3B876B}

%% DIFFCOEFF PKG SETTING

\difdef { l } { dn } { style = d^ }
\difdef { l } {} {outer-Rdelim = \,}
\difdef {f, s} {D}{op-symbol = \mathrm{D}}



\renewcommand{\contentsname}{Contents}
\AtBeginDocument{\addtocontents{toc}{\protect\thispagestyle{fancy}}}

%\renewcommand{\cftchapfont}{\normalsize}


