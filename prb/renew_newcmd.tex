%% ========== %%
%%  Tcolorbox %%
%% ========== %%
\tcbset{
  dashedbox/.style={
    breakable,
    enhanced,
    frame hidden,
    borderline={0.3pt}{0pt}{blue!50!white,dashed},
    %fontupper = \fontsize{9pt}{8pt}\selectfont,
    %before upper = {\everymath{\scriptstyle}},
    colback = white,
    arc = 0mm,
    outer arc = 0mm,
    boxsep = 0.5pt,
    leftrule=-6pt,
    rightrule=-6pt
  }
}
\tcbset{
  leftsolid/.style={
    breakable,
    enhanced,
    colback = white,
    frame hidden,
    %arc = 0mm,
    %outer arc = 0mm,
    boxsep = 0.5pt,
    leftrule=4pt,
    rightrule=-6pt,
    borderline north={0.3pt}{0pt}{orange!30!white},
    borderline east={0.3pt}{0pt}{orange!30!white},
    borderline south={0.3pt}{0pt}{orange!30!white},
    borderline west={10pt}{0pt}{orange!10!white}
  }
}
\tcbset{
  leftsolid2/.style={
    breakable,
    enhanced,
    colback = white,
    frame hidden,
    %arc = 0mm,
    %outer arc = 0mm,
    boxsep = 0.5pt,
    leftrule=4pt,
    rightrule=-6pt,
    borderline north={0.3pt}{0pt}{blue!30!white},
    borderline east={0.3pt}{0pt}{blue!30!white},
    borderline south={0.3pt}{0pt}{blue!30!white},
    borderline west={10pt}{0pt}{blue!10!white}
  }
}
%% ================ %%
%% renewcmd, newcmd %%
%% ================ %%


%\renewcommand{\baselinestretch}{0.85}
\renewcommand{\headrulewidth}{0pt} %% headerの線の太さ.0ptで線無し
\renewcommand{\figurename}{Fig.}
\renewcommand{\tablename}{Table}



\newcommand{\mctext}[1]{\mbox{\textcircled{\scriptsize{#1}}}}
\newcommand{\ctext}[1]{\textcircled{\scriptsize{#1}}}
\newcommand{\ds}{\displaystyle}
\newcommand{\comb}[2]{\left(\begin{matrix}#1\\#2\end{matrix}\right)}
\newcommand{\hs}{\hspace}
\newcommand{\qq}{\hspace{1em}}
\newcommand{\qqtext}[1]{\hspace{1em}\text{#1}\hspace{0.5em}}
\newcommand{\vs}{\vspace}
\newcommand{\emphvs}{\vspace{1em}\notag\\}
\newcommand{\ora}{\overrightarrow}
\newcommand{\oramr}[1]{\overrightarrow{\mathrm{#1}}}
\newcommand{\ol}{\overline}
\newcommand{\tri}{\triangle}
\newcommand{\mr}{\mathrm}
\newcommand{\mb}{\mathbb}
\newcommand{\mrvec}[1]{\overrightarrow{\mathrm{#1}}}
\newcommand{\itvec}{\overrightarrow}
%\newcommand{\bs}{\boldsymbol}
\newcommand{\bs}{\symbfit}%%symbfでdefine
\newcommand{\bsup}{\symbfup}

\newcommand{\ra}{\rightarrow}
\newcommand{\Ra}{\Rightarrow}
\newcommand{\lra}{\longrightarrow}
\newcommand{\Lra}{\Longrightarrow}
\newcommand{\la}{\leftarrow}
\newcommand{\La}{\Leftarrow}
\newcommand{\lla}{\longleftarrow}
\newcommand{\Lla}{\Longleftarrow}
\newcommand{\lr}{\leftrightarrow}
\newcommand{\llr}{\longleftrightarrow}
\newcommand{\Llr}{\Longleftrightarrow}
\newcommand{\mqty}[1]{\begin{matrix}#1\end{matrix}}
\newcommand{\avg}[1]{\left\langle{#1}\right\rangle}

\newcommand{\eval}[1]{\left.#1\right|}
\newcommand{\order}[1]{\mathcal{O}\ab(#1)}
\newcommand{\gr}{\nabla}
\newcommand{\di}{\nabla\cdot}
\newcommand{\ro}{\nabla\times}

\DeclareMathOperator{\GL}{GL}
\DeclareMathOperator{\SL}{SL}
\DeclareMathOperator{\diag}{diag}
\DeclareMathOperator{\tr}{tr}
\DeclareMathOperator{\sinc}{sinc}

\renewcommand{\i}{\mathrm{i}}
\newcommand{\e}{\mathrm{e}}


\difdef { l } { dn } { style = d^ }
\difdef{l}{}{outer-Rdelim = \,}

\renewcommand{\contentsname}{Contents}
\AtBeginDocument{\addtocontents{toc}{\protect\thispagestyle{fancy}}}

%\renewcommand{\cftchapfont}{\normalsize}

