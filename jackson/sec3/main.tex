%! TeX root = main.tex
%% kind : preamble file
%% tex type : LuaLaTeX
%% created 2024.10.24
%% modified 2025.01.03
%% paper size : A4
%% font en : CMU
%% font ja : Hiragino Sans W2 / W4
%% HiraMinProN-w3
%% using physics2 pkg, backgroundcolor = True/cream

\documentclass{ltjsarticle}

\usepackage{amsmath, amssymb, amsthm}
\usepackage{tikz}

\usepackage{ascmac} % boxnote, itembox, mask, screen, shadebox
\usepackage{comment}
\usepackage{empheq}
\usepackage[shortlabels, inline]{enumitem}
\usepackage{fancybox} % doublebox, ovalbox, fbox, sbox
\usepackage{fancyhdr}
    %%\pagestyle{fancy}
    %%\[r, l, c]head{hoge}
    %%\[r, l, c]foot{hoge}


\usepackage{here}
\usepackage{lastpage}
\usepackage{listings, jvlisting} % code insert // jvlisting→jlistingに移行すべきとも
    %%\lstset{}以下でコードの出力方法について指定

\usepackage{stmaryrd} % 特殊記号類
\usepackage[listings]{tcolorbox}
\usepackage{titlesec} % change title format
     %% example %% 

     %%\titleformat{\section}[display]{\bfseries\gtfamily}{}{10pt}{%
     %%\begin{tcolorbox}[%
     %%    enhanced,%
     %%    colback=white,colbacktitle=white,colframe=black,coltitle=black,%
     %%    boxrule=1.5pt,sharp corners,%
     %%    borderline={1.5pt}{3pt}{black},%
     %%    ]%
     %%    \centering Section \thesection \\
     %%    \LARGE #1%
     %%\end{tcolorbox}%
     %%}
     %%\titleformat{\section}[block]{
     %%   \normalfont\bfseries\filcenter
     %%   }{\fbox{\itshape\thesection}}{1em}{}
     %%\titleformat{\section}[hang]{\large\bfseries}
     %%  {\colorbox{blue!20}{\thesection}}{10.5pt}{}%
     %%  [\color{blue!20}{\titlerule}]

%% color setting
     %% \definecolor{mycream}{HTML}{FEFAEA}
     %% \definecolor{myvio}{HTML}{05208C}

     %% \titleformat{\section}[hang]{\large\bfseries}
     %% {\colorbox{myvio}{\color{mycream}\thesection}}{1\zw}{}%
     %% [{\titlerule[0.5pt]}]
    
     %% \usepackage{pagecolor}
	 %% \pagecolor{mycream}
	 %% \color{myvio}



\usepackage{ulem} %% uline, uuline, uwave, sout, xout
\usepackage{url} %% url
\usepackage{verbatim} %%\verb|hoge| or verbatim env. option[*] : space visible
\usepackage{wrapfig}
\usepackage{xcolor}
%\usepackage{luatexja-ruby}

%\usepackage{minipage}
\usepackage{varwidth} %% minipageより自然な幅?

\usepackage{siunitx}
\usepackage{physics2}
  \usephysicsmodule{ab}
  %\usephysicsmodule{ab.braket} %% conflict w/ braket
  \usephysicsmodule{braket}
  \usephysicsmodule{diagmat}
  \usephysicsmodule{doubleprod}
  \usephysicsmodule{xmat}

\usepackage[ISO]{diffcoeff}
  \difdef {f, s} {D}
  {op-symbol = \mathrm{D}}


\usepackage{hyperref}
\hypersetup{%
colorlinks=true,%
}

\numberwithin{equation}{section} %% equation numbering setting

%% ============ %%
%% font setting %%
%% ============ %%
%% series 1
%%\usepackage[no-math, deluxe, expert]{luatexja-preset}
%%\setmainfont[Ligatures=TeX]{EB Garamond Regular}[
%%  BoldFont = {EB Garamond Bold},
%%  ItalicFont = {EB Garamond Italic},
%%  BoldItalicFont = {EB Garamond Bold Italic}
%%  ]
\usepackage[math-style=TeX]{unicode-math}
\usepackage{mathtools}

%%\setmathfont{Garamond-Math.otf}%[StylisticSet={7,9}]
%%\setmainjfont{TsukuhouMincho Regular}[
%%  BoldFont = {Zen Old Mincho Bold}
%%  ]
%%\setsansjfont{Hiragino Sans GB W3}


%% series 2
\usepackage[no-math,deluxe,expert]{luatexja-preset}
%\setmainjfont{Hiragino Sans W2}[
%  BoldFont = {Hiragino Sans W4}
%]
\setmainjfont{HiraMinProN-W3}[
	BoldFont = {Hiragino Sans W4}
]

\usepackage{luatexja-otf}
\usepackage{utfsym}

%% ====================== %%
%% titleformat (titlesec) %%
%% ====================== %%

%\titleformat*{\section}{\large\bfseries\sffamily}
%\titleformat*{\subsection}{\large\bfseries\sffamily}
%\titleformat*{\subsubsection}{\normalsize\bfseries\sffamily}

\titleformat*{\section}{\normalsize\bfseries}
\titleformat*{\subsection}{\normalsize\bfseries}
\titleformat*{\subsubsection}{\normalsize\bfseries}
%%
\renewcommand{\thesection}{\arabic{section}}
\renewcommand{\thesubsection}{\thesection.\arabic{subsection}}
\renewcommand{\thesubsubsection}{\thesection.\arabic{subsection}.\arabic{subsubsection}}

\titlespacing{\section}{0pt}{7pt}{5pt}[4cm]
\titlespacing{\subsection}{0pt}{5pt}{5pt}
\titlespacing{\subsubsection}{0pt}{3pt}{5pt}

\theoremstyle{definition}
\newtheorem{dfn}{Definition}[section]
\newtheorem{prop}[dfn]{Proposition}
\newtheorem{lem}[dfn]{Lemma}
\newtheorem{thm}[dfn]{Theorem}
\newtheorem{cor}[dfn]{Corollary}
\newtheorem{rem}[dfn]{Remark}
\newtheorem{fact}[dfn]{Fact}

%% ============= %%
%% tcbuselibrary %%
%% ============= %%
  \tcbuselibrary{most}
  \tcbset{colback=white,colframe=blue!75!black}
  \newtcbtheorem[number within = section]{example}{Example}{
    enhanced,
    %colback = white
    colframe = blue!50!white,
    %fonttitle=\bfseries,
    breakable = true
  }{thm}
  %enhanced, breakable=true, colframe=blue!75!black, colback=blue!5!white, title={Exemple}, arc = 0mm}
  %{colback=green!5,colframe=green!35black,fonttitle=\bfseries}{th}

  \newtcbtheorem[number within = section]{exercise}{Exercise}{
  enhanced, 
  breakable=true,
  colframe=yellow!25!green
  %colback=white
  }{thm}

  \newtcbtheorem[use counter from=example]{theorem}{Theorem}{
  enhanced, 
  breakable=true, 
  colframe=green!30!blue
  %colback=white
  }{thm}

  \newtcbtheorem[use counter from=example]{proposition}{Proposition}{
  enhanced, 
  breakable=true, 
  colframe=orange!75!black,
  %colback=white
  }{thm}

  \newtcbtheorem[use counter from=example]{definition}{Definition}{
  enhanced, 
  breakable=true, 
  colframe=magenta!30!green,
  %colback=white
}{thm}

  \newtcbtheorem[use counter from=example]{remark}{Remark}{
  enhanced,
  breakable=true, 
  colframe=red!80!white, 
  %colback=white
}{thm}



%% =============================== %%
%% import renewcommand, newcommand %%
%% =============================== %%

%% main.tex のファイルからの相対パスを示すか,絶対パスを示す
%% ================ %%
%% renewcmd, newcmd %%
%% ================ %%

%\renewcommand{\baselinestretch}{0.85}
\renewcommand{\headrulewidth}{0pt} %% headerの線の太さ.0ptで線無し
\renewcommand{\figurename}{Fig.}
\renewcommand{\tablename}{Table}



\newcommand{\mctext}[1]{\mbox{\textcircled{\scriptsize{#1}}}}
\newcommand{\ctext}[1]{\textcircled{\scriptsize{#1}}}
\newcommand{\ds}{\displaystyle}
\newcommand{\comb}[2]{\left(\begin{matrix}#1\\#2\end{matrix}\right)}
\newcommand{\hs}{\hspace}
\newcommand{\qq}{\hspace{1em}}
\newcommand{\vs}{\vspace}
\newcommand{\emphvs}{\vspace{1em}\notag\\}
\newcommand{\ora}{\overrightarrow}
\newcommand{\oramr}[1]{\overrightarrow{\mathrm{#1}}}
\newcommand{\ol}{\overline}
\newcommand{\tri}{\triangle}
\newcommand{\mr}{\mathrm}
\newcommand{\mb}{\mathbb}
\newcommand{\mrvec}[1]{\overrightarrow{\mathrm{#1}}}
\newcommand{\itvec}{\overrightarrow}
%\newcommand{\bs}{\boldsymbol}
\newcommand{\bs}{\symbfit}%%symbfでdefine
\newcommand{\bsup}{\symbfup}

\newcommand{\ra}{\rightarrow}
\newcommand{\Ra}{\Rightarrow}
\newcommand{\lra}{\longrightarrow}
\newcommand{\Lra}{\Longrightarrow}
\newcommand{\la}{\leftarrow}
\newcommand{\La}{\Leftarrow}
\newcommand{\lla}{\longleftarrow}
\newcommand{\Lla}{\Longleftarrow}
\newcommand{\lr}{\leftrightarrow}
\newcommand{\llr}{\longleftrightarrow}
\newcommand{\Llr}{\Longleftrightarrow}
\newcommand{\mqty}[1]{\begin{matrix}#1\end{matrix}}
\newcommand{\avg}[1]{\left\langle{#1}\right\rangle}

\newcommand{\eval}[1]{\left.#1\right|}
\newcommand{\order}[1]{\mathcal{O}\ab(#1)}
\newcommand{\gr}{\nabla}
\newcommand{\di}{\nabla\cdot}
\newcommand{\ro}{\nabla\times}

\DeclareMathOperator{\GL}{GL}
\DeclareMathOperator{\SL}{SL}
\DeclareMathOperator{\diag}{diag}
\DeclareMathOperator{\tr}{tr}

\renewcommand{\i}{\,\mathrm{i}}
\newcommand{\e}{\,\mathrm{e}}



%%% ================ %%
%% renewcmd, newcmd %%
%% ================ %%

%\renewcommand{\baselinestretch}{0.85}
\renewcommand{\headrulewidth}{0pt} %% headerの線の太さ.0ptで線無し
\renewcommand{\figurename}{Fig.}
\renewcommand{\tablename}{Table}



\newcommand{\mctext}[1]{\mbox{\textcircled{\scriptsize{#1}}}}
\newcommand{\ctext}[1]{\textcircled{\scriptsize{#1}}}
\newcommand{\ds}{\displaystyle}
\newcommand{\comb}[2]{\left(\begin{matrix}#1\\#2\end{matrix}\right)}
\newcommand{\hs}{\hspace}
\newcommand{\qq}{\hspace{1em}}
\newcommand{\vs}{\vspace}
\newcommand{\emphvs}{\vspace{1em}\notag\\}
\newcommand{\ora}{\overrightarrow}
\newcommand{\oramr}[1]{\overrightarrow{\mathrm{#1}}}
\newcommand{\ol}{\overline}
\newcommand{\tri}{\triangle}
\newcommand{\mr}{\mathrm}
\newcommand{\mb}{\mathbb}
\newcommand{\mrvec}[1]{\overrightarrow{\mathrm{#1}}}
\newcommand{\itvec}{\overrightarrow}
%\newcommand{\bs}{\boldsymbol}
\newcommand{\bs}{\symbfit}%%symbfでdefine
\newcommand{\bsup}{\symbfup}

\newcommand{\ra}{\rightarrow}
\newcommand{\Ra}{\Rightarrow}
\newcommand{\lra}{\longrightarrow}
\newcommand{\Lra}{\Longrightarrow}
\newcommand{\la}{\leftarrow}
\newcommand{\La}{\Leftarrow}
\newcommand{\lla}{\longleftarrow}
\newcommand{\Lla}{\Longleftarrow}
\newcommand{\lr}{\leftrightarrow}
\newcommand{\llr}{\longleftrightarrow}
\newcommand{\Llr}{\Longleftrightarrow}
\newcommand{\mqty}[1]{\begin{matrix}#1\end{matrix}}
\newcommand{\avg}[1]{\left\langle{#1}\right\rangle}

\newcommand{\eval}[1]{\left.#1\right|}
\newcommand{\order}[1]{\mathcal{O}\ab(#1)}
\newcommand{\gr}{\nabla}
\newcommand{\di}{\nabla\cdot}
\newcommand{\ro}{\nabla\times}

\DeclareMathOperator{\GL}{GL}
\DeclareMathOperator{\SL}{SL}
\DeclareMathOperator{\diag}{diag}
\DeclareMathOperator{\tr}{tr}

\renewcommand{\i}{\,\mathrm{i}}
\newcommand{\e}{\,\mathrm{e}}





\pagestyle{fancy}  
\fancyhead[L]{\rightmark}
\fancyhead[C]{}  
\fancyhead[R]{\textbf{\thepage}}  
\fancyfoot{}
\renewcommand{\headrulewidth}{0.1pt}
\renewcommand{\contentsname}{Contents}

\begin{document}  
\begin{screen}
  \centering
  J. D. Jackson, Classical Electrodynamics

  Section 3: Boundary-Value Problems in Electrostatics: II ノート
\end{screen}
\begin{flushright}  
  @crutont
\end{flushright}

\setcounter{section}{3}
\tableofcontents
\hrulefill
\newpage

\subsection{Laplace Equation in Spherical Coordinates}
\subsection{Legendre Equation and Legendre Polynomials}
\subsection{Boundary-Value Problems with Azimuthal Symmetry}
\subsection{Behavior of Fields in a Conical Hole or Near a Sharp Point}
\subsection{Associated Legendre Functions and the Spherical Harmonics \texorpdfstring{$Y_{lm}(\theta, \phi)$}{}}
\subsection{Addition Theorem for Spherical Harmonics}
\subsection{Laplace Equation in Cylindrical Coordinates; Bessel Functions}
\subsection{Boundary-Value Problems in Cylindrical Coordinates}
\subsection{Expansion of Green Functions in Spherical Coordinates}
\subsection{Solution of Potential Problems with the Spherical Green Function Expansion}
\subsection{Expansion of Green Functions in Cylindal Coordinates}
円筒座標系でGreen関数の満たすべき式;
\begin{gather}
   \label{eq:Greenfunc_in_cylindar}
  \gr^2_x G(\bs{x}, \bs{x}') = -\frac{4\pi}{\rho} \delta(\rho - \rho') \delta(\phi - \phi') \delta (z - z')
\end{gather}
$\phi$と$z$に関する部分のデルタ関数を
\begin{gather}
  \delta(\phi - \phi') = \frac{1}{2\pi} \sum_{m=-\infty}^{\infty} \e^{\i m(\phi - \phi')}\\
\delta(z - z') = \frac{1}{2\pi} \int_\infty^\infty \dl{k} \e^{\i k(z-z')}  = \frac{1}{\pi} \int_0^\infty \dl{k}\cos\ab[k(z-z')]
\end{gather}
として,Green関数が
\begin{gather}  
  G(\bs{x}, \bs{x}') = \frac{1}{2\pi^2} \sum_{m=-\infty}^{\infty} \int_0^\infty \dl{k} \e^{\i m(\phi-\phi')} \cos \ab[k(z - z')] g_m(k; \rho, \rho')
\end{gather}
として表されるとする.

Green関数の式\eqref{eq:Greenfunc_in_cylindar}に代入して,丁寧に計算をすると,$g_m$の部分についての関係式が得られる;
\begin{gather}
  \label{eq:Greenfunc_cylindar_radius}
  \frac{1}{\rho} \diff**{\rho}{\ab(\rho\diff{g_m}{\rho})} - \ab(k^2 + \frac{m^2}{\rho^2})g_m = -\frac{4\pi}{\rho} \delta (\rho - \rho')
\end{gather}
$\rho \neq \rho'$のときは右辺がゼロになり,これはmodified Bessel functionである.一般解は$I_m(k\rho)$と$K_m(k\rho)$の線型結合で表される.

$\psi_1(k\rho)$と$\psi_2(k\rho)$がそれぞれ,$\rho < \rho'$,$\rho > \rho'$での境界条件を満たす,$I_m, K_m$の線型結合で表される(線型独立な)解であるとすると,Green関数の対称性($\rho$と$\rho'$の入れ替えに関して同じ関数を与えること)より,
\begin{gather}
  \label{eq:Greenfunc_cylindar_radius_psi}
  g_m(k; \rho, \rho') = \psi_1(k\rho_<)\psi_2(k\rho_>)
\end{gather}
で表すことができる.

$\rho = \rho'$での接続条件を考えると,
\begin{gather}  
  \eval{\diff{g_m}{\rho}}_{\rho = \rho' +0} - \eval{\diff{g_m}{\rho}}_{\rho = \rho' - 0} = - \frac{4\pi}{\rho'}
\end{gather}
となる.
これに,\eqref{eq:Greenfunc_cylindar_radius_psi}を代入すると,
\begin{gather}
  k W\ab[\psi_1(k\rho), \psi_2(k\rho)] = -\frac{4\pi}{\rho} \\
  \label{eq:wronskian_cylindar}
  \text{i.e.}\qq W\ab[\psi_1(x), \psi_2(x)] = -\frac{4\pi}{x}
\end{gather}
となる.ただし,$W\ab[\psi_1, \psi_2]$は$\psi_1$と$\psi_2$のWronskianであり,
\begin{gather}
  W\ab[\psi_1, \psi_2] = \psi_1 \psi_2' - \psi_1'\psi_2
\end{gather}
で与えられる.

\eqref{eq:Greenfunc_cylindar_radius}はStrum-Liouville型の微分方程式;
\begin{gather}
  \diff**{x}{\ab[p(x) \diff{y}{x}]} + g(x) y = 0
\end{gather}
である.
この方程式の線型独立な2つの解のWronskianは$1/p(x)$に比例する形でかけることが知られている.
このことを認めることにする.\footnote{いつかメモにしたい}

いま,境界面がない自由空間を考える.
$g_m$は$\rho = 0$で有限かつ$\rho \to \infty$でゼロになるので,(係数を$\psi_1$に取り込むことにすると)$\psi_1(k\rho) = AI_m(k\rho)$および,$\psi_2(k\rho) = K_m(k\rho)$と表される.係数$A$はWronskianの条件から定める.

Wronskianは$1/\rho$に比例しているが,これはどの$\rho$についても成立している.いま,$\rho \gg 1$の極限で考えることにすると,($\rho \ll 1$の極限で考えても良い)
modified Bessel functionの漸近形
\begin{gather}  
  \begin{dcases}
    I_m(x) \sim \frac{1}{\sqrt{2\pi x}}\e^x\\
    K_m(x) \sim \sqrt{\frac{\pi}{2x}}\e^{-x}
  \end{dcases}
\end{gather}
を代入して計算すれば,
\begin{gather}
  W \ab[I_m(x), K_m(x)] = -\frac{1}{x}
\end{gather}
となる,したがって,\eqref{eq:wronskian_cylindar}と係数を比較することにより,$A = 4\pi$を得る.

さて,自由空間におけるGreen関数は$G(\bs{x}, \bs{x}') = 1 / \ab|\bs{x} - \bs{x}'|$であったから,この結果より,
\begin{gather}
  \frac{1}{\ab|\bs{x} - \bs{x}'|} = \frac{2}{\pi} \sum_{m=-\infty}^{\infty} \int_0^\infty \dl{k} \e^{\i m(\phi - \phi')} \cos\ab[k(z - z')] I_m(k\rho_<) K_m(k\rho_>)
\end{gather}
これは,
実関数だけを用いて表すことができて,
\begin{multline}
  \frac{1}{\ab|\bs{x} - \bs{x}'|} = \frac{4}{\pi} \int_0^\infty \dl{k} \cos\ab[k(z - z')] \\
  \times \ab\{\frac{1}{2}I_0(k\rho_<)K_0(k\rho_>) + \sum_{m=1}^\infty \cos\ab[m(\phi - \phi')] I_m(k\rho_<) K_m(k\rho_>)\}
\end{multline}
と書くことができる.

$\bs{x}' = 0$としたとき,$m \geq 1$に対して,$I_m(0) = 0$であるから,$m = 0$の項だけが残り,($I_m(z) \sim (z/2)^m $ if $z \ll 1$である.)
\begin{gather}
  \label{eq:cosK0}
  \frac{1}{\sqrt{\rho^2 + z^2}} = \frac{2}{\pi} \int_0^\infty \dl{k} \cos(kz) K_0(k\rho)
\end{gather}
となる.
また,$\rho^2 \to R^2 = \rho^2 + (\rho')^2 - 2 \rho \rho' \cos(\phi - \phi')$と置き直すと, このとき,$|\bs{x} - \bs{x}'(z'=0)| = \sqrt{R^2 + z^2}$であるから,
\begin{gather}
  \frac{1}{\sqrt{R^2 + z^2}} = \frac{4}{\pi} \int_0^\infty \dl{k} \cos(kz) \ab\{ \frac12 I_0(k\rho_<)K_0(k\rho_>) + \sum_{m=1}^{\infty}\cos\ab[m(\phi - \phi')]I_m(k\rho_<) K_m(k\rho_>)\}
\end{gather}
これと\eqref{eq:cosK0}を比較して,
\begin{multline}
  \label{eq:k0sqrt}
  K_0\ab(k\sqrt{\rho^2 + (\rho')^2 - 2 \rho \rho'\cos(\phi - \phi')}) \\
  = I_0(k\rho_<)K_0(k\rho_>) + 2 \sum_{m=1}^{\infty}\cos\ab[m(\phi - \phi')]I_m(k\rho_<) K_m(k\rho_>)
\end{multline}
を得る.ここで,$k\to 0$の極限を考える.$z \ll 1$のとき,
\begin{align}
  I_m(z) &\sim \frac{1}{\Gamma(m+1)} \ab(\frac{z}{2})^m\\
  K_m(z) &\sim
  \begin{dcases}
    -\log\ab(\frac{z}2) - 0.5772\ldots \qq (m=0)\\
    \frac{\Gamma(m)}{2}\ab(\frac{2}{z})^m\qq (m\neq 0)
  \end{dcases}
\end{align}
で漸近形が与えられるので,これを\eqref{eq:k0sqrt}に代入して計算をすると,
\begin{gather}
  \log\ab(\frac{1}{\rho^2 + (\rho')^2 - 2 \rho \rho' \cos(\phi - \phi')}) = 2\log\ab(\frac{1}{\rho_>}) + 2 \sum_{m=1}^{\infty}\frac{1}{m}\ab(\frac{\rho_<}{\rho_>})^m\cos[m(\phi - \phi')]
\end{gather}
これは二次元極座標系におけるGreen関数の表式である.(問題2.17)

%%2025-01-24 書き終え 3.11

\subsection{Eigenfunction Expansions for Green Functions}

\subsection{Mixed Boundary Conditions; Conducting Plane with a Circular Hole}

\subsection{Problems}
章末問題の手の付け方,略解とかを書きたい.
\begin{enumerate}[label=\fbox{\textbf{\thesection.\arabic*}}]
  \item 方位角対称なポテンシャルをLegendre多項式で展開したときの一般形
    \begin{gather}
      \label{eq:potential_in_legendre}
      \Phi(r, \theta) = \sum_{l=0}^\infty (A_l r^l + B_l r^{-(l+1)})P_l(\cos\theta)
    \end{gather}
    に代入して境界条件から各係数を決定する.
    Legendre多項式について成り立つ関係:
    \begin{gather}
      P_n(0) = (-1)^{n/2} 2^{-n} \comb{n}{n/2} \qq (n : \text{even})
    \end{gather}
    を使えば良い.
  \item 
    \begin{enumerate}[(a)]  
      \item \eqref{eq:potential_in_legendre}
        のように展開して,表面電荷密度による境界条件を考えることで係数を決定する.Legendre多項式の直交関係式:
        \begin{gather}
          \int_0^1 \dl{(\cos\theta)} P_{l'}(\cos\theta) P_l(\cos\theta) = \frac{2}{2l+1} \delta_{ll'}
        \end{gather}
        および,Legendre多項式の基本的な式:
        \begin{gather}
        P_l(-x) = (-1)^l P_l(x)
        \end{gather}
        を用いれば良い.
      \item $\bs{E} = - \gr \Phi$を用いて原点での電場を計算する.$\hat{\bs{z}} = \cos\theta\hat{\bs{r}} - \sin\theta \hat{\bs{\theta}}$を用いると結果を$\hat{\bs{z}}$のみで表すことができる.
      \item (2)では$\beta = \pi - \alpha$として$\beta \to 0$の極限を考えるとよい.
  \end{enumerate}
  \item この問題は議論の余地がある.
    ちょっと後回しにする.
  \item
    \begin{enumerate}[(a)]  
      \item  
    ポテンシャルを球面調和関数を用いた展開で表す:

    \begin{gather}
      \Phi(r, \theta, \phi) = \sum_{l=0}^\infty \sum_{m=l}^{l} \ab(A_{lm}r^l + B_{lm}r^{-(l+1)}) Y_{lm}(\theta,\phi)
    \end{gather}
    $r=a$の球面上で境界条件から各係数を定める.
    計算の途中で,
    \begin{gather}
      V(\phi) = \left\{
      \begin{aligned}
        +V &\qq \text{if}\qq \frac{\pi}{n}\cdot 2j \leq \phi \leq \frac{\pi}{n}\cdot(2j+1)\\
        -V &\qq \text{if}\qq \frac{\pi}{n}\cdot (2j+1) \leq \phi \leq \frac{\pi}{n}\cdot(2j+2)\\
      \end{aligned}
    \right. \qq (\text{for}\ j = 0, 1, \cdots, n-1)
    \end{gather}
    を用いて,積分
    \begin{gather}
      \int_0^{2\pi} \dl{\phi} V(\phi) \e^{-\i m\phi}
    \end{gather}
    の計算があらわれるが,$m \neq 0$のときを考えると,
    \begin{gather}
      \int_0^{2\pi} \dl{\phi} V(\phi) \e^{-\i m\phi} = -\frac{\i V}{m}\ab[1 - \exp\ab(-\i \frac{m}{n}\pi)]^2 \sum_{j=0}^{n-1} \exp\ab(-\i\frac{m}{n}2j\pi)
    \end{gather}
    となる.$m = 0, \pm 2n, \pm 4n, \ldots$のときは$\exp\ab(-\i(m/n)\pi) = 1$となるので,積分はゼロ.
    いま,$m$は整数であるから,残りとして考えるのは$m = \pm n, \pm 3n, \ldots$である.
    このとき,
    \begin{gather}
      \exp\ab(-\i \frac{m}{n}2j \pi) = 1
    \end{gather}
    であるから,これらをまとめて
    \begin{gather}
      \int_0^{2\pi} \dl{\phi} V(\phi) e^{-\i m \phi} = 
        \begin{dcases}
          - \i\frac{4Vn}{m} & \text{if}\qq m = \pm n, \pm 3n, \ldots\\
           \qq 0 &\text{otherwise}
        \end{dcases}
    \end{gather}
    と書ける.
  \item (a)の結果を用いて具体的に計算を進めていくだけであるが,Jackson §3.3 eq. (3.36)との比較の時には座標軸の取り方に気をつける必要がある.具体的には,$\cos\theta' = \sin\theta \sin \phi$とすれば良い.
    \end{enumerate}
  \item Jackson eq. (3.70)の式を$r, a$でそれぞれ微分して,差を考えて$\dl{\Omega'}V(\theta' , \phi')$で積分をすれば良い.
  \item ポテンシャルが具体的に計算できるので,Jackson eq. (3.70)を用いて,球面調和関数で展開して,和を考えれば良い.この問題は易しい.
  \item 前問と同じように考えれば良い.
  \item $\log(\csc\theta) = \log(1/\sin\theta)$をLegendre多項式で展開する.このときに積分
    \begin{gather}
      \int \dl{x} \log(1 - x^2) = (1+x)\log(1+x) - (1-x)\log(1-x) - 2x
    \end{gather}
    を用いる.普通に$\log$積分をしてしまうと発散してしまうことに注意.この積分は知らないとキビシイかも.
  \item 円筒座標系でのLaplace方程式
    \begin{gather}
      \frac{1}{\rho}\diffp**{\rho}{\ab(\rho \diffp{\Phi}{\rho})} + \frac{1}{\rho^2}\diffp[2]{\Phi}{\phi} + \diffp[2]{\Phi}{z} = 0
   \end{gather}
   を$z = 0, L$での境界条件に注意して解けば良い.
 \item 前問の結果を用いればよい.(b)での極限を考える時は,
   \begin{gather}
     I_\nu(z) \sim \frac{1}{\nu !} \ab(\frac{z}{2})^\nu + \order{z^{\nu+1}} \qq \text{if} \qq z \ll 1
   \end{gather}
   を用いると良い.また,三角関数の無限和を求める時は,$\exp$の形にして無限級数として和を求めてその実部・虚部を求めると見通しが良い.また,$\log$の虚部は$\arg$で与えられることに注意.
 \item %3.11
 \item %3.12
 \item %3.13
 \item %3.14
  線電荷密度を求めて,それを体積電荷密度で書くと
  \begin{gather}
    \rho_\mr{c} (\bs{x}) = \frac{3Q}{8\pi d^3} \frac{d^2 - r^2}{r^2} \ab[\delta(\cos\theta - 1) + \delta(\cos\theta + 1)]
  \end{gather}
  と表すことができる.Jackson eq. (3.125)のGreen関数を用いて球内部のポテンシャルを求めればよい.$r \gtrless d$で積分の計算が異なるが,丁寧に計算をすれば良い.
 \item %3.15
 \item %3.16
   Besselの微分方程式
   \begin{gather}
     \diff[2]{J_\nu(k\rho)}{\rho} + \frac{1}{\rho} \diff{J_\nu(k\rho)}{\rho} + \ab(k^2 - \frac{\nu^2}{\rho^2})J_\nu(k\rho)  = 0
   \end{gather}
   から出発して,部分積分,$k, k'$の入れ替えをして差を考えると,
   \begin{gather}
     (k^2 - (k')^2) \int_0^\infty \dl{\rho}\rho J_\nu(k\rho)J_\nu(k'\rho) = \eval{\ab[\rho J_\nu(k\rho)\diff{J_\nu(k'\rho)}{\rho} - \rho J_\nu(k'\rho)\diff{J_\nu(k\rho)}{\rho}]}_{\rho=0}^{\rho=\infty}
   \end{gather}
   となる.$\rho = \infty$の場合を考える.$\rho = R$として,$R \to \infty$を考えることにする.Bessel関数の漸近形を用いて,
   \begin{multline}  
     \eval{\ab[\rho J_\nu(k\rho)\diff{J_\nu(k'\rho)}{\rho} - \rho J_\nu(k'\rho)\diff{J_\nu(k\rho)}{\rho}]}_{\rho = R} \\
     =\frac{1}{\pi}\frac{1}{\sqrt{kk'}} \ab[\frac{-1}{k+k'}\cos[(k+k')R - \nu \pi] + \frac1{k'-k}\sin\ab[(k'-k)R]]_{R \to \infty}
   \end{multline}
   と書き直すことができる.

   デルタ関数は,$\sinc$関数を用いて
   \begin{gather}
     \lim_{\epsilon \to 0} \delta_\epsilon(x) = \lim_{\epsilon \to 0} \frac{\sin(x/\epsilon)}{x / \epsilon} \frac{1}{\pi \epsilon} = \delta(x)
   \end{gather}
   として近似することができる.
   \begin{gather}
     \lim_{R \to \infty} \frac{\cos\ab[(k+k')R - \nu \pi]}{k+k'} \propto \delta_{1/R}\ab((k+k') - \ab(\nu - \frac{1}{2}\frac{\pi}{R})) = 0\\
     \lim_{R\to \infty}
   \end{gather}

 \item %3.17
 \item %3.18
 \item %3.19
 \item %3.20
 \item %3.21
\end{enumerate} 

\end{document}

